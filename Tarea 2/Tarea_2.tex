\documentclass[fleqn]{article}
\usepackage{amsmath, amssymb}
\usepackage[spanish]{babel}
\usepackage{parskip, nopageno}
\usepackage[a4paper, margin = 2.1cm]{geometry}
\renewcommand{\theenumii}{\roman{enumii}}   %Redefinir los simbolos de lista con minúsculas romana, (es necesario cargar el idioma inglés)
\renewcommand{\labelenumii}{{\theenumii}.}   %Poner un punto despúes del símbolo de lista

\begin{document}
    %Portada
    \begin{titlepage}
        \centering
        {\Huge \textbf{Universidad Autónoma del Estado de México}\par}
        \vspace{0.9cm}

        {\Huge \textbf{Facultad de Ciencias}\par}
        \vspace{0.9cm}

        {\Huge \textbf{Licenciatura en Matemáticas}\par}
        \vspace{1.8cm}

        {\huge \textbf{Álgebra Lineal}\par}
        \vspace{0.7cm}

        {\huge \textbf{Profesora:}\par}
        \vspace{0.3cm}
        {\huge \textsl{Socorro López Olvera}}\par
        \vspace{0.7cm}

        {\huge \textbf{Tarea 2}\par}
        \vspace{0.7cm}

        {\huge \textbf{Alumnos:}\par}
        \vspace{0.3cm}
        {\huge \textsl{Peña Mateos Jesús Jacob}}\par
        \vspace{1.5mm}
        {\huge \textsl{Santana Reyes Osmar Dominique}}\par
        {\huge \textsl{Gallegos Torres Gonzalo}}\par
        \vspace{1.8cm}

        {\huge \textbf{Semestre: 2022B}\par}
        \vspace{0.7cm}
        \vfill
        \raggedleft{\LARGE Fecha de Entrega:}\par
    \end{titlepage}

    \begin{enumerate}
        \bfseries
        %Ejercicio 1
        \item Diga si las siguientes proposiciones son falsas o verdaderas y justifique su respuesta.
    
        \begin{enumerate}
            %Inciso i
            \item La matriz identidad es una matriz triangular superior. \par
            
            \normalfont
            \textbf{Af:} La matriz identidad es una matriz triangular superior. \par

            \hspace{2.7mm} \textbf{Dem.} \par

            Sea $ I_n = [a_{ij}] $ la matriz identidad. Por definición, 
            \begin{equation*}
                I_n = 
                \begin{pmatrix}
                    1      & 0      & \cdots & 0\\ 
                    0      & 1      & \cdots & 0\\
                    \vdots & \vdots & \ddots & \vdots\\
                    0      & 0      & \cdots & 1
                \end{pmatrix}
            \end{equation*}
            Ya que $ a_{ij} = 0 \;\, \forall \, i > j $, entonces $ I_n $ es una matriz triangular superior, por definición. $ \blacksquare $ \par

            %Inciso ii
            \bfseries
            \item Si $ V $ es un $ K $ \textsl{- espacio vectorial}, $ W \leq V $ y $ U \leq W $ entonces $ U \leq V $. \par
            
            \normalfont
            \textbf{Af:} $ U \leq V $. \par

            \hspace{2.7mm}\textbf{Dem.} \par

            Como $ W \leq V $ y $ U \leq W $ entonces $ W \subseteq V $ y $ U \subseteq W $. De esta menera, como $ U \subseteq W $ y $ W \subseteq V $, entonces $ U \subseteq V $. \par

            Luego, sea $ \overline{0} \in V $ el neutro aditivo. Como $ W \leq V $ entonces $ \overline{0} \in W $, por el inciso \textit{i} de la proposición 5. Y ya que $ U \leq W $ entonces $ \overline{0} \in U $, por el inciso \textit{i} de la proposición 5. \par

            Después, como $ W \leq V $ entonces la adición es cerrada en $ W $, por el inciso \textit{ii} de la proposición 5. Y ya que $ U \leq W $ entonces la adición es cerrada en $ U $, por el inciso \textit{ii} de la proposición 5. \par

            Y finalmente, como $ W \leq V $ entonces el producto por escalar es cerrado en $ W $, por el inciso \textit{iii} de la proposición 5. Y ya que $ U \leq W $ entonces el producto por escalar es cerrado en $ U $, por el inciso \textit{iii} de la proposición 5. \par

            En conclusión, como $ \overline{0} \in U $, la adición es cerrada en $ U $ y el producto por escalar es cerrado en $ U $, entonces $ U \leq V $, por la proposición 5. $ \blacksquare $ \par

            %Inciso iii
            \bfseries
            \item Si $ A $ es una matriz con $ tr(A) = 0 $, entonces $ A $ es matriz antisimétrica. \par
            
            \normalfont
            \textbf{Af:} $ A $ no es matriz antisimétrica. \par

            \hspace{2.7mm}\textbf{Dem.} \par

            Considerando la matriz
            \begin{equation*}
                A =
                \begin{pmatrix}
                    1  & 5  & 7\\
                    -5 & -1 & 9\\
                    -7 & -9 & 0
                \end{pmatrix}
            \end{equation*}
            Por definición, la traza de A es $ tr(A) = 1 + (-1) + 0 = 0 $. Así, $ tr(A) = 0 $. \par
            Luego,
            \begin{equation*}
                A^t =
                \begin{pmatrix}
                    1 & -5 & -7\\
                    5 & -1 & -9\\
                    7 & 9  & 0
                \end{pmatrix}
                \quad \text{y} \quad -A =
                \begin{pmatrix}
                    -1 & -5 & -7\\
                    5  & 1  & -9\\
                    7  & 9  & -0
                \end{pmatrix}
                =
                \begin{pmatrix}
                    -1 & -5 & -7\\
                    5  & 1  & -9\\
                    7  & 9  & 0
                \end{pmatrix}
            \end{equation*}
            Como $ A^t \neq -A $ entonces la matriz $ A $ no es transpuesta. Por lo tanto, no es cierto que si $ A $ es una matriz con $ tr(A) = 0 $, entonces $ A $ es matriz antisimétrica. $ \blacksquare $ \par

        \end{enumerate}

        %Ejercicio 2
        \item Sean $ n \in \mathbb{N} $, $ K $ un campo, $ A \in M_n[K] $ y $ A^t $ su transpuesta.
        
        \begin{enumerate}
            %Inciso i
            \item Demuestra que $ A + A^t $ es una matriz simétrica.
            
            %Inciso ii
            \item Sea $ U = \left \{A \in M_n [K] \; \middle \arrowvert \; A \text{ es antisimétrica} \right \} \subseteq M_n [K] $. Demuestre que $ U \leq M_n[K] $.
            
            %Inciso iii
            \item ¿Cuántas entradas diferentes puede tener una matriz simétrica de orden $ n $? Argumente su respuesta.
        \end{enumerate}

        %Ejercicio 3
        \item Sea $ V $ un $ K $-espacio vectorial y $ W_1, W_2, \, \ldots \, , W_n \leq V $ entonces:
        \begin{enumerate}
            \item $	W_1 + W_2 + \cdots + W_n \leq V $. \par
            \normalfont
            \hspace{2.7mm} \textbf{Dem.} \par

            Pd. $ \overline{0} \in W_1 + W_2 + \cdots + W_n $. \par

            Como $ W_1, W_2,  \, \ldots \, , W_n \leq V $, entonces $ \overline{0} \in W_1, W_2, \, \ldots \, , W_n $ para cada $ i=1, \, \ldots \, ,n $. Así,
            \begin{equation*}
                \underbrace{\overline{0} + \overline{0} + \cdots + \overline{0}} \in W_1 + W_2 + \cdots + W_n
            \end{equation*}
            \vspace{0.5mm} n veces
            Es decir,
            \begin{equation*}
                \overline{0} \in W_1 + W_2 + \cdots + W_n
            \end{equation*}
            Pd. Cerradura de la suma. \par

            Sean $ v, w \in W_1 + W_2 + \cdots + W_n $, entonces existen $ v_1, w_1 \in W_1 $, $ v_2, w_2 \in W_2 $, $ \, \ldots \, $, $ v_n, w_n \in W_n $ tales que
            \begin{equation*}
                v = v_1 + v_2 + \cdots + v_n \quad \quad \quad y \quad \quad \quad
                w = w_1 + w_2 + \cdots + w_n
            \end{equation*}
            Entonces
            \begin{equation*}
                v + w = \left( v_1 + v_2 + \cdots + v_n \right) + \left( w_1 + w_2 + \cdots + w_n \right)
            \end{equation*}
            Y conmutando y asociando se tiene que 
            \begin{equation*}
                v + w = \left( v_1 + w_1 \right) + \left( v_2 + w_2 \right) + \cdots + \left( v_n + w_n \right)
            \end{equation*}
            De donde $ \left( v_1 + w_1 \right) \in W_1 $, $ \left( v_2 + w_2 \right) \in W_2 $, $ \, \ldots \, $, $ \left( v_n + w_n \right) \in W_n $ pues $ W_1, W_2, \, \ldots \, , W_n \leq V $. Así pues
            \begin{equation*}
                v + w \in W_1 + W_2 + \cdots + W_n
            \end{equation*}
            Pd. Cerradura del producto por escalar. \par
            Sean $ v \in W_1 + W_2 + \cdots + W_n $ y $ \delta \in K $, entonces existen $ v_1 \in W_1 $, $ v_2 \in W_2 $, $ \, \ldots \, $, $ v_n \in W_n $ tales que
            \begin{equation*}
                v = v_1 + v_2 + \cdots + v_n
            \end{equation*}
            Entonces
            \begin{align*}
                \delta v =& \, \delta \left( v_1 + v_2 + \cdots + v_n \right) \\
                =& \, \delta v_1 + \delta v_2 + \cdots + \delta v_n 
            \end{align*}
            De donde $ \delta v_1 \in W_1 $, $ \delta v_2 \in W_2 $, $ \, \ldots \, $, $ \delta v_n \in W_n $ pues $ W_1, W_2, \, \ldots \, , W_n \leq V $. Así,
            \begin{align*}
                & \delta v_1 + \delta v_2 + \cdots + \delta v_n \in W_1 + W_2 + \cdots + W_n \\
                & \text{Es decir,} \\
                & \delta v \in W_1 + W_2 + \cdots + W_n
            \end{align*}
            $ \therefore W_1 + W_2 + \cdots + W_n \leq V $.

            \item $ W_i \subseteq W_1 + W_2 + \cdots + W_n; \, \forall \, i = 1,  \, \ldots \, , n $. \par
            \hspace{2.7mm} \textbf{Dem.} \par

            Sea $ u_i \in W_i $. Como $ \overline{0} \in W_1, W_2, \ldots, W_n $, pues $ W_1, W_2, \ldots, W_n \leq V $, entonces
            \begin{equation*}
                \overline{0} + \overline{0} + \ldots + u_i + \ldots + \overline{0} \in W_1 + W_2 + \cdots + W_i + \cdots + W_n
            \end{equation*}
            Es decir,
            \begin{equation*}
                u_i \in W_1 + W_2 + \cdots + W_i + \cdots + W_n
            \end{equation*}
            $ \therefore W_i \subseteq W_1 + W_2 + \cdots + W_n $. $ \blacksquare $
        \end{enumerate}
        
        %Ejercicio 4
        \item Sea $ V $ un $ K $ \textsl{- espacio vectorial} y $ W \subseteq V $ no vacío. Demuestre que $ W \leq V $, si y solo si se cumple:
        
        \begin{enumerate}
            %Inciso i
            \item $ u - z \in W $ para cualesquiera $ u, \, z \in W $.
            
            %Inciso ii
            \item $ \lambda u \in W $, $ \forall \, \lambda \in K $ y $ \forall \, u \in W $.
        \end{enumerate}

        %Ejercicio 5
        \item Sean $ V = \mathbb{R}^3 $, un $ \mathbb{R} $ \textsl{- espacio vectorial}, con las operaciones usuales de suma y producto por escalar, $ W_1 $ y $ W_2 $ subconjuntos de $ V $, definidos como, $ W_1 = \left \{(a,b,c) \in \mathbb{R}^3 \middle \arrowvert 3a - b + 4c = 0 \right \} $ y $ W_2 = \left \{(a,b,c) \in \mathbb{R}^3 \middle \arrowvert b=-a, 2c=a \right \} $ ¿Es $ V = W_1 \, \oplus \, W_2 $? Demuéstrelo o dé un contraejemplo de la propiedad que no se cumpla.
        
        \normalfont
        \textbf{Af:} $ V = W_1 \, \oplus \, W_2 $. \par

        \hspace{2.7mm}\textbf{Dem.} \par

        \begin{minipage}[c]{0.7cm}
            \vspace{-1.7cm} P.d. 
        \end{minipage} \begin{minipage}[b]{9cm}
            \begin{enumerate}
                \item[a)] $ W_1 $ y $ W_2 $ son subespacios vectoriales de $ V $. 
                \item[b)] $ V = W_1 + W_2 $
                \item[c)] $ W_1 \cap W_2 = \left \{\overline{0} \right \} $
            \end{enumerate}
        \end{minipage}
        
        \begin{enumerate}
            \item[a)] $ W_1 $ y $ W_2 $ son subespacios vectoriales de $ V $. \par
            Como $ \overline{0} = (0,0,0) $ es el neutro aditivo en $ V $, entonces $ \overline{0} \in W_1 $ pues $ 3(0) - 0 + 4(0) = 0 - 0 + 0 = 0 $. \par

            Luego, sean $ (a_1,b_1,c_1), (d_1,e_1,f_1) \in W_1 $ entonces $ 3a_1 - b_1 + 4c_1 = 0 $ y $ 3d_1 - e_1 + 4f_1 = 0 $. Por definición de suma en $ \mathbb{R}^3 $ se tiene que \par

            $ (a_1,b_1,c_1) + (d_1,e_1,f_1) = (a_1+d_1, \, b_1+e_1, \, c_1+f_1) $ \par

            Y como
            \begin{align*}
                3(a_1+d_1) - (b_1+e_1) + 4(c_1+f_1) =& \; (3a_1 + 3d_1) + (- b_1 - e_1) + (4c_1 + 4f_1) \\
                =& \; (3a_1 - b_1 + 4c_1) + (3d_1 - e_1 + 4f_1) \\
                =& \; 0 + 0 \\
                & (\text{pues } 3a_1 - b_1 + 4c_1 = 0 \text{ y } 3d_1 - e_1 + 4f_1 = 0) \\
                =& \; 0
            \end{align*}
            entonces $ (a_1,b_1,c_1) + (d_1,e_1,f_1) \in W_1 $.

            Después, sean $ \lambda \in \mathbb{R} $ y $ (a,b,c) \in W_1 $ entonces $ 3a - b + 4c = 0 $. Luego, por definición de producto por escalar, se tiene que \par

            $ \lambda (a,b,c) = (\lambda a, \lambda b, \lambda c) $ \par

            Y como
            \begin{align*}
                3\lambda a - \lambda b + 4 \lambda c =& \lambda (3a - b + 4c) \\
                =& \lambda (0) &(\text{pues } 3a - b + 4c = 0) \\
                =& 0
            \end{align*}
            entonces $ \lambda (a,b,c) \in W_1 $ \par

            Como $ \overline{0} \in W_1 $, la suma y el producto por escalar son cerrados en $ W_1 $, entonces $ W_1 \leq V $, por la Proposición 5.
            
            Ahora, se tiene que $ \overline{0} \in W_2 $ pues $ 0 = -0 $ y $ 2(0) = 0 $.

            Luego, sea $ (a_2,b_2,c_2) \in W_2 $ entonces $ b_2 = -a_2 $ y $ 2c_2 = a_2 $. Y sea $ (d_2,e_2,f_2) \in W_2 $ entonces $ e_2 = -d_2 $ y $ 2f_2 = d_2 $. Por definición de suma en $ \mathbb{R}^3 $ se tiene que \par

            $ (a_2,b_2,c_2) + (d_2,e_2,f_2) = (a_2 + d_2, \, b_2 + e_2, \, c_2 + f_2) $ \par

            Y como
            \begin{align*}
                b_2 + e_2 =& \; -a_2 + (-d_2) &(\text{pues } b_2 = -a_2 \text{ y } e_2 = -d_2) \\
                =& \; -(a_2 + d_2) \\\\
                2(c_2 + f_2) =& \; 2c_2 + 2f_2 \\
                =& \; a_2 + d_2 &(\text{pues } 2c_2 = a_2 \text{ y } 2f_2 = d_2)
            \end{align*}
            entonces $ (a_2,b_2,c_2) + (d_2,e_2,f_2) \in W_2 $. \par

            Después, sean $ \lambda \in \mathbb{R} $ y $ (d,e,f) \in W_2 $ entonces $ e = -d $ y $ 2f = d $. Luego, por definición de producto por escalar, se tiene que \par

            $ \lambda (d,e,f) = (\lambda d, \lambda e, \lambda f) $ \par

            Y como
            \begin{align*}
                \lambda e =& \lambda (-d)  &(\text{pues } e = -d) \\
                =& - (\lambda d) &(\text{inciso 3 de la proposición 3}) \\\\
                2 (\lambda f) =& \lambda (2f) \\
                =& \lambda d &(\text{pues } 2f = d) 
            \end{align*}
            entonces $ \lambda (d,e,f) \in W_2 $ \par

            Como $ \overline{0} \in W_2 $, la suma y el producto por escalar son cerrados en $ W_2 $, entonces $ W_2 \leq V $, por la Proposición 5. \par

            \item[b)] $ V = W_1 + W_2 $ \par
            \begin{enumerate}
                %Contenido
                \item[$ \subseteq $]] Sea $ (a,b,c) \in \mathbb{R}^3 $. P.d. $ (a,b,c) \in W_1 + W_2 $. \par
                
                Sean $ u = \left( \displaystyle \frac{3a + b - 4c}{6}, \frac{3a + 5b + 4c}{6}, \frac{-3a + b + 8c}{12} \right) $ y $ \displaystyle v = \left(\frac{3a - b + 4c}{6}, \frac{-3a + b - 4c}{6}, \frac{3a - b + 4c}{12} \right) $. Como
                \begin{align*}
                    3 \left(\frac{3a + b - 4c}{6} \right) - \frac{3a + 5b + 4c}{6} + 4 \left(\frac{-3a + b + 8c}{12} \right) = \; & \frac{9a + 3b - 12c}{6} \; + \\ & \frac{-3a - 5b - 4c}{6} \; + \\ & \frac{-6a + 2b + 16c}{6} \\
                    = \; & 0
                \end{align*}
                entonces $ u \in W_1 $. Y también se tiene que
                \begin{align*}
                    \frac{-3a + b - 4c}{6} = - \frac{3a - b + 4c}{6} \quad y \\
                    2 \left( \frac{3a - b + 4c}{12} \right) = \frac{3a - b + 4c}{6}
                \end{align*}
                De esta manera, $ v \in W_2 $. Luego, $ u + v \in W_1 + W_2 $, pero
                \raggedright \begin{align*}
                    u + v = \; & \left( \displaystyle \frac{3a + b - 4c}{6}, \frac{3a + 5b + 4c}{6}, \frac{-3a + b + 8c}{12} \right) + \\
                    & \left( \frac{3a - b + 4c}{6}, \frac{-3a + b - 4c}{6}, \frac{3a - b + 4c}{12} \right) \\
                    = \; & \left( \displaystyle \frac{3a + b - 4c}{6} + \frac{3a - b + 4c}{6} \, , \; \frac{3a + 5b + 4c}{6} + \frac{-3a + b - 4c}{6} \, , \; \frac{-3a + b + 8c}{12} + \frac{3a - b + 4c}{12} \right) \\
                    = \; & \left( \displaystyle \frac{6a}{6} \, , \; \frac{6b}{6} \, , \; \frac{12c}{12} \right) \\
                    = \; & \left(a,b,c \right)
                \end{align*}
                Así, $ (a,b,c) \in W_1 + W_2 $. \par
                $ \therefore V \subseteq W_1 + W_2 $.

                %Contiene
                \item[$ \supseteq $]] Como $ W_1, W_2 \leq V $ entonces, por el Teorema 12, $ W_1 + W_2 \leq V $ y por definición de subespacio vectorial $ W_1 + W_2 \subseteq V $.
            \end{enumerate}
            $ \therefore V = W_1 + W_2 $ \par
            
            \item[c)] $ W_1 \cap W_2 = \left \{\overline{0} \right \} $
            \begin{enumerate}
                \item[$ \subseteq $]] Sea $ (a,b,c) \in W_1 \cap W_2 $. P.d. $ (a,b,c) \in \left \{\overline{0} \right \} $.

                Como $ (a,b,c) \in W_1 \cap W_2 $ entonces $ (a,b,c) \in W_1 $ y $ (a,b,c) \in W_2 $. Ya que $ (a,b,c) \in W_1 $ entonces 
                \begin{equation}
                    3a - b + 4c = 0
                    \label{eq:incisou}
                \end{equation}
                Asimismo, como $ (a,b,c) \in W_2 $ entonces 
                \begin{equation}
                    b = -a 
                    \label{eq:incisod}
                \end{equation} 
                y \par
                $ \quad \quad \, 2c = a $
                \begin{align}
                    \Longrightarrow c = \frac{a}{2}
                    \label{eq:incisot}
                \end{align}
                Sustituyendo (\ref{eq:incisod}) y (\ref{eq:incisot}) en (\ref{eq:incisou}) se tiene que
                \begin{align*}
                    & 3a - (-a) + 4 \left( \frac{a}{2} \right) = 0 \\
                    & \Longrightarrow 3a + a + 2a = 0 \\
                    & \Longrightarrow 6a = 0 \\
                    & \Longrightarrow a = 0
                \end{align*}
                Así, por (\ref{eq:incisod}), se tiene que $ b = -a = -0 = 0 $. Y por (\ref{eq:incisot}) $ c = \displaystyle \frac{a}{2} = \frac{0}{2} = 0 $. De esta manera, $ (a,b,c) = (0,0,0) = \overline{0} $. \par
                $ \therefore (a,b,c) \in \left \{\overline{0} \right \} $.

                \item[$ \supseteq $]] Como $ W_1, W_2 \leq V $ entonces $ \overline{0} $ pertenece a $ W_1 $ y a $ W_2 $. Así, $ \overline{0} \in W_1 \cap W_2 $. \par 

                $ \therefore \left\{\overline{0} \right\} \subseteq W_1 \cap W_2 $
            \end{enumerate}
            $ \therefore W_1 \cap W_2 = \left \{\overline{0} \right \} $
        \end{enumerate}
        $ \therefore V = W_1 \, \oplus \, W_2 $. $ \blacksquare $
    \end{enumerate}
\end{document}
