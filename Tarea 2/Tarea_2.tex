\documentclass[fleqn]{article}
\usepackage{amsmath, amssymb}
\usepackage[spanish, english]{babel}
\usepackage{parskip}
\usepackage[a4paper, margin=3cm]{geometry}
\renewcommand{\theenumii}{\textit{\roman{enumii}}}   %Redefinir los simbolos de lista con minúsculas romana, (es necesario cargar el idioma inglés)
\renewcommand{\labelenumii}{{\theenumii}.}   %Poner un punto despúes del símbolo de lista

\begin{document}
    %Portada
    \begin{titlepage}
        \centering
        {\Huge \textbf{Universidad Autónoma del Estado de México}\par}
        \vspace{0.9cm}

        {\Huge \textbf{Facultad de Ciencias}\par}
        \vspace{0.9cm}

        {\Huge \textbf{Licenciatura en Matemáticas}\par}
        \vspace{1.8cm}

        {\huge \textbf{Álgebra Lineal}\par}
        \vspace{0.7cm}

        {\huge \textbf{Profesora:}\par}
        \vspace{0.3cm}
        {\huge \textsl{Socorro López Olvera}}\par
        \vspace{0.7cm}

        {\huge \textbf{Tarea 2}\par}
        \vspace{0.7cm}

        {\huge \textbf{Alumnos:}\par}
        \vspace{0.3cm}
        {\huge \textsl{Peña Mateos Jesús Jacob}}\par
        \vspace{1.5mm}
        {\huge \textsl{Santana Reyes Osmar Dominique}}\par
        {\huge \textsl{Gallegos Torres Gonzalo}}\par
        \vspace{1.8cm}

        {\huge \textbf{Semestre: 2022B}\par}
        \vspace{0.7cm}
        \vfill
        \raggedleft{\LARGE Fecha de Entrega:}\par
    \end{titlepage}

    \begin{enumerate}
        %Ejercicio 1
        \item Diga si las siguientes proposiciones son falsas o verdaderas y justifique su respuesta.
    
        \begin{enumerate}
            %Inciso i
            \item La matriz identidad es una matriz triangular superior. \par
            Af! La matriz identidad es una matriz triangular superior. \par
            \hspace{2.7mm}Dem.\par
            Por definicion, $ I_n = [a_{ij}] \;\; \forall \, i = j $



            %Inciso ii
            \item Si $ V $ es un $ K $ \textsl{- espacio vectorial}, $ W \leq V $ y $ U \leq W $ entonces $ U \leq V $.
            
            %Inciso iii
            \item Si $ A $ es una matriz con $ tr(A) = 0 $, entonces $ A $ es matriz antisimétrica. 
        \end{enumerate}

        %Ejercicio 2
        \item Sean $ n \in \mathbb{N} $, $ K $ un campo, $ A \in M_n[K] $ y $ A^t $ su transpuesta.
        
        \begin{enumerate}
            %Inciso i
            \item Demuestra que $ A + A^t $ es una matriz simétrica.
            
            %Inciso ii
            \item Sea $ U = \left \{A \in M_n [K] \; \middle \arrowvert \; A \text{ es antisimétrica} \right \} \subseteq M_n [K] $. Demuestre que $ U \leq M_n[K] $.
            
            %Inciso iii
            \item ¿Cuántas entradas diferentes puede tener una matriz simétrica de orden $ n $? Argumente su respuesta.
        \end{enumerate}

        %Ejercicio 3
        \item Demuestre el corolario 13.
        
        %Ejercicio 4
        \item Sea $ V $ un $ K $ \textsl{- espacio vectorial} y $ W \subseteq V $ no vacío. Demuestre que $ W \leq V $, si y solo si se cumple:
        
        \begin{enumerate}
            %Inciso i
            \item $ u - z \in W $ para cualesquiera $ u, \, z \in W $.
            
            %Inciso ii
            \item $ \lambda u \in W $, $ \forall \, \lambda \in K $ y $ \forall \, u \in W $.
        \end{enumerate}

        %Ejercicio 5
        \item Sean $ V = \mathbb{R}^3 $, un $ \mathbb{R} $ \textsl{- espacio vectorial}, con las operaciones usuales de suma y producto por escalar, $ W_1 $ y $ W_2 $ subconjuntos de $ V $, definidos como, $ W_1 = \left \{(a,b,c) \in \mathbb{R}^3 \middle \arrowvert 3a-b+4c=0 \right \} $ y $ W_2 = \left \{(a,b,c) \in \mathbb{R}^3 \middle \arrowvert b=-a, 2c=a \right \} $ ¿Es $ V = W_1 \, \oplus \, W_2 $? Demuéstrelo o dé un contraejemplo de la propiedad que no se cumpla.
        
    \end{enumerate}

\end{document}
